\section{Comparison to selected papers}

This section will contextualize G-Retriever in the field of RAG by comparing it to selected papers, highlighting similarities and key differences between different systems.

\subsection{Retrieval-Augmented Generation for    Knowledge-Intensive NLP Tasks}

The original RAG paper by \citet{rag} introduced the RAG framework, which allows pre-trained language models to access external knowledge sources during generation.
Like G-Retriever, RAG uses Similarity of embeddings to retrieve relevant information from external sources, but instead of text-based graphs, it uses chunked Wikipedia articles as knowledge source.
Unlike G-Retriever, where the retrieval part is frozen and does not change during training, RAG uses a retriever model that is fine-tuned jointly with the generative model.
Both papers focus on question answering scenarios.

\subsection{REALM: Retrieval-Augmented Language Model Pre-Training}

REALM \cite{realm} is also one of the very early papers of RAG.
The authors use masked language modelling to pre-train a language model on a large corpus of text and then fine-tune it for Open-Domain Question Answering.
Like RAG, REALM uses Wikipedia articles as external data source, not graphs.

A key difference to the G-Retriever model is that REALM uses a retriever model that is fine-tuned jointly with the generative model, gradients calculated for all documents.
The use of salient span masking also sets it apart from G-Retriever.

\subsection{In-Context Retrieval-Augmented Language Models}

The core contribution of this paper is the proof that most benefits of RAG can be had with a very simple setup, where documents are retrieved and prepended to the query, which can then be given to any off-the-shelf LLM, without the need to perform any training at all \cite{in-context}.
Other papers usually modify the LLM architecture in some way to to retrieval augmentation, with various parts being frozen or trainable, depending on the specific paper.

Using unmodified LLMs has the advantage of not requiring training resources and enabling faster deployment, for example by using off-site LLMs via API-Access.



\subsection{Internet-Augmented Dialogue Generation}

In \cite{komeili-etal-2022-internet}, the authors describe a dialog focused system that is grounded with information retrieved from the internet at generation time.
To do this, their system generates a query based on its context, which they supply to a search engine such as Bing.
The documents that are retrieved by this search engine are then added to the context to generate the actual model response.

The real-time usage of the internet enables this approach to access the latest information available on the internet, similar to how a human would probably search for information.

Apart from the general concept of retrieving information and using it to improve the generation results, this approach has little in common with G-Retriever. While the authors of G-retriever, \citet{g-retriever}, mention in passing the option for a "Chat-with-your-graph" system, they do not explore this in their paper, and the resulting G-Retriever is not really a dialog focused system.

\subsection{Retrieval-Augmented Retriever Multimodal Language Modeling}

RA-CM3 is a retrieval augmented multi-modal model, where both the knowledge sources and the output can be text or images \cite{multimodal}.

Like many RAG approaches, RA-CM3 uses Maximum Inner Product Search to find relevant documents from its pool of candidates, but a special multi-modal encoder like CLIP is necessary to allow both images and text to be used.

The authors show that their approach gives better generation results, while using less training resources at the same time.
Additionally, they explore other use cases such as controlled image generation and image editing.

While G-Retriever is evaluated on spatial reasoning tasks, it can not not use images as input or output, instead needing descriptions of the scene as textual graph as knowledge source.

\subsection{Retrieval meets Long Context Large Language Models}

How do retrieval augmentation and long context windows for LLMs compare, and how does a combination of both perform?
Those are the central questions that \citet{Xu2023RetrievalML} set out to answer in this 2024 paper.

The authors evaluate their system with a dataset focused on question answering, where some questions require multi-hop reasoning, similar to the data in G-Retriever.

Results show that retrieval augmentation can bring the performance of models with small context windows on par with models much larger context, while reducing inference time.

\subsection{Benchmarking Large Language Models in Retrieval-Augmented Generation}

The authors of \citet{benchmarking} identify several key challenges that RAG models in general face. Those are \textbf{noise robustness} (the ability to generate correct responses when not all of the supplied documents are relevant to the answer), \textbf{negative rejection} (the ability to decline to answer if not enough information is known), \textbf{information integration} (the ability to aggregate information from multiple documents) and \textbf{counterfactual robustness} (the ability detect conflicting information).

While G-Retriever is not among the systems that have been examined for this paper, these could still be relevant.
One exception might be information integration, since G-Retriever only uses a single (sub-) graph to condition on.



\subsection{G-Retriever}

How does G-Retriever compare to these models?

The main difference G-Retriever has compared to others is its focus on text-based graphs as knowledge sources, and the subgraph construction step that other approaches do not need.

It also performs prompt tuning, which is not common in other RAG models. Its retrieval component is not trained, which is a difference to some, but not all models. The actual language model is also frozen, with the only trainable parts being the graph encoder and projection layer.
While the authors mention the possibility of using the model in a dialog system, in this paper they use G-Retriever for question answering, which is a very popular use case for RAG systems.
