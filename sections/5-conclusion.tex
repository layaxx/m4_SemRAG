\section{Conclusion}

The field of retrieval-augmented generation presents a promising approach for enhancing the quality of generated text by incorporating information from retrieved documents while maintaining lower resource requirements compared to scaling up language models.

While most RAG systems focus on unstructured text sources \textemdash{} such as news articles, scientific papers, or Wikipedia \textemdash{} G-Retriever distinguishes itself by leveraging structured graphs as its knowledge source.
This approach enables the system to utilize both the structure of the graph and its textual content, potentially enhancing reasoning capabilities.
However, it also introduces the challenge of designing a more complex retrieval mechanism compared to standard text-based approaches.

One promising yet underexplored avenue in RAG research is the use of retrieved information to explain the model's reasoning process.
Enhancing transparency in this way could provide users with more contextual background, facilitate fact-checking, and improve the detection of potential errors in the model's output.
Future work could further investigate this aspect, potentially contributing to more interpretable and trustworthy AI systems.
