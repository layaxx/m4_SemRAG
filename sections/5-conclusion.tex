\section{Conclusion}

The field of RAG is a promising approach to improve the quality of generated text by incorporating information from retrieved documents while keeping resource requirements lower than scaling up the language model.
While most systems focus on unstructured text, such as news articles, scientific paper or Wikipedia, G-Retriever uses structured knowledge graphs as knowledge source.
This allows the system to make use of the information contained in the structure itself, as well as the text content, but also introduces the need for a more complex retrieval mechanism.

One possibility that is made available by RAG but seems under-explored in the literature, is the option to use the retrieved information to explain the model's reasoning.
This could provide users with more background information and enable them to fact-check and spot mistakes in the model's generation process.

