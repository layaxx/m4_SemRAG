\begin{abstract}
    The research field of Retrieval-Augmented Generation (RAG), a subfield of Natural Language Processing (NLP), combines the strengths of retrieval-based and generation-based systems and has gained significant traction in recent years.
    This paper provides an overview of key concepts, contributions, and approaches in RAG, with a particular focus on the graph-based G-Retriever model. It examines the similarities and differences between G-Retriever and selected other models, highlighting its unique characteristics and potential advantages.
\end{abstract}
