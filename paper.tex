% This must be in the first 5 lines to tell arXiv to use pdfLaTeX, which is strongly recommended.
\pdfoutput=1
% In particular, the hyperref package requires pdfLaTeX in order to break URLs across lines.

\documentclass[11pt]{article}

% Remove the "review" option to generate the final version.
\usepackage{acl_mod}

% Standard package includes
\usepackage{times}
\usepackage{latexsym}

% For proper rendering and hyphenation of words containing Latin characters (including in bib files)
\usepackage[T1]{fontenc}
% For Vietnamese characters
% \usepackage[T5]{fontenc}
% See https://www.latex-project.org/help/documentation/encguide.pdf for other character sets

% This assumes your files are encoded as UTF8
\usepackage[utf8]{inputenc}

% This is not strictly necessary, and may be commented out,
% but it will improve the layout of the manuscript,
% and will typically save some space.
\usepackage{microtype}

% This package helps printing tables more nicely.
\usepackage{booktabs}

% Uncomment and modify for custom submission date
% \renewcommand{\subdate}{October 3, 2022}

%--------------------------------------------------------------------------------------------------
%--------------------------------------------------------------------------------------------------
% Modify title, author information, module name here

\def\paperTitle{G-Retriever in the context of the RAG research field}
\def\authors{Yannick Lang}
\def\matriculationNumber{1995498}
\def\moduleName{DS-SemRAG-M}

%--------------------------------------------------------------------------------------------------
%--------------------------------------------------------------------------------------------------


\title{\paperTitle}

\author{\authors \\
  Matriculation Number: \matriculationNumber \\
  \moduleName \\
  Faculty of Information Systems and Applied Computer Sciences\\
  Otto-Friedrich-University of Bamberg}

\begin{document}
\maketitle
\begin{abstract}
    The research field of Retrieval Augmented Generation (RAG) is a subfield of Natural Language Processing (NLP) that combines the strengths of retrieval-based and generation-based models.
    This paper aims to give an overview over key concepts, contributions and approaches, focusing on the G-Retriever model and its similarities with and difference to other approaches.
\end{abstract}


{\footnotesize
  \tableofcontents}

While Large Language Models (LLMs) have demonstrated strong performance across various Natural Language Processing (NLP) tasks, including question answering, they are prone to hallucinations and rely on static, potentially outdated information \cite{Marcus2020TheND}.
LLMs encode knowledge within their model weights \cite{petroni-etal-2019-language}, but their capacity is constrained by model size, and their knowledge remains fixed after training. Updating or modifying this knowledge is computationally expensive and requires full or partial retraining \cite{realm}.

Retrieval-Augmented Generation (RAG), a subfield of NLP, addresses these limitations by integrating retrieval-based and generation-based approaches.
By retrieving relevant information from external knowledge sources and incorporating it into the generation process, RAG enhances output accuracy and adaptability.

\section{Retrieval Augmented Generation}

This section will give an overview of the research field of Retrieval Augmented Generation (RAG), with a focus on its inception and key contributions leading up to the G-Retriever paper.

RAG is a subfield of Natural Language Processing that has gained traction since its inception in 2020. It combines the strengths of retrieval-based and generation-based models, aiming to improve the quality of generated text by incorporating information from retrieved documents.

The foundational work in this field was done by \cite{rag}, who introduced the RAG framework.
The main idea behind this paper was to allow pre-trained language models to access external knowledge sources during generation, thereby allowing the model to condition on information retrieved from external sources, in addition to  the knowledge embedded into its weight during the training process.
The authors show that such external knowledge sources can be updated, expanded, and refined independently of the model, which allows for more up-to-date and diverse information without the need to expend the resources associated with training large language models.

\section{G-Retriever}

This section will introduce the 2024 G-Retriever model and its key concepts, contributions, and approaches.

\subsection{Motivation}
G-Retriever is a model that applies the RAG approach to text-based graphs.
It was introduced in 2024 by \cite{g-retriever} and focuses on text-based graphs, i.e. graphs where both nodes and edges are associated with text labels, as knowledge sources.
Such graphs are common in many domains, for example knowledge graphs, social networks, and scene graphs.
Since these graphs can be quite large with only a small fraction of the nodes and edges relevant to a given generation task, G-Retriever aims to retrieve the most relevant subgraph for a given generation task.

\subsection{Architecture}

The process consists of four steps: (1) Indexing, (2) retrieval, (3) subgraph construction, and (4) generation.

In the indexing step, the node and edge attributes of the graph are turned into vector representations via a pre-trained language model and stored in a nearest neighbor data structure.

In the retrieval step, the query is encoded using the same pre-trained language model, and the top-k most relevant nodes and edges are retrieved based on the similarity between the query and the node and edge representations.
Because one of the key benefits of graphs over unstructured text are the relation between entities, an additional step is necessary that other, document-based approaches do not need:
Using the relevance scores of nodes and edges we just calculated, we want to find an optimal subgraph from the original graph that we can give as context to the model.
This is necessary, because the relevant nodes and edges might be from different parts of the graph and not even be connected to each other, therefore missing information that might be helpful.

In the subgraph construction step, the retrieved nodes and edges are used to construct a subgraph that contains as much relevant information as possible while including as little unnecessary nodes and edges as possible.
This is achieved with a modified version of the Price-collecting Steiner Tree (PCST) algorithm.
Modifications are necessary, because the original PCST problem assumes that all information is contained in the nodes, which are assigned prices, and edges only have non-negative costs, i.e. cannot add any value to a solution.
For this task, edges are associated with text attributes which may be relevant to the query, therefore edges edges can also to have prices in addition to costs.
The authors show, that this modified problem is equivalent to the original PCST problem.
If a given edge has a price that is lower than its cost, this can be treated as a reduced cost.
Since PCST does not allow for negative edge costs, another approach needs to be taken if the price exceeds the cost.
Then, a virtual node is introduced that connects the two nodes connected by the original edge, and the price of the virtual node is set to the difference between the price and the cost of the edge.

Once this optimal subgraph has been constructed, it is supplied to the generative model for the generation step. The subgraph is used in two ways: (1) encoded via a graph encoder, scaled to the correct dimension via a projection layer and supplied to the LLM used in the generation process, and (2) prepended to the query in a textualized form.

\subsection{Evaluation and Results}



\section{Comparison to selected papers}

This section will contextualize G-Retriever in the field of RAG by comparing it to selected papers, highlighting similarities and key differences.

\subsection{Retrieval-Augmented Generation for    Knowledge-Intensive NLP Tasks}

The original RAG paper by \citet{rag} introduced the RAG framework, which allows pre-trained language models to access external knowledge sources during generation.
Like G-Retriever, RAG uses Similarity of embeddings to retrieve relevant information from external sources, but instead of text-based graphs, it uses chunked Wikipedia articles as knowledge source.
Unlike G-Retriever, where the retrieval part is frozen and does not change during training, RAG uses a retriever model that is fine-tuned jointly with the generative model.
Both papers focus on question answering scenarios.

\subsection{REALM: Retrieval-Augmented Language Model Pre-Training}

REALM \cite{realm} is also one of the very early papers of RAG.
The authors use masked language modelling to pre-train a language model on a large corpus of text and then fine-tune it for Open-Domain Question Answering.
Like RAG, REALM uses Wikipedia articles as external data source, not graphs.

A key difference to the G-Retriever model is that REALM uses a retriever model that is fine-tuned jointly with the generative model, gradients calculated for all documents.
The use of salient span masking also sets it apart from G-Retriever.

\subsection{In-Context Retrieval-Augmented Language Models}

The core contribution of this paper is the proof that most benefits of RAG can be had with a very simple setup, where documents are retrieved and prepended to the query, which can then be given to any off-the-shelf LLM, without the need to perform any training at all \cite{in-context}.
Other papers usually modify the LLM architecture in some way to to retrieval augmentation, with various parts being frozen or trainable, depending on the specific paper.

Using unmodified LLMs has the advantage of not requiring training resources and enabling faster deployment, for example by using off-site LLMs via API-Access.



\subsection{Internet-Augmented Dialogue Generation}

In \cite{komeili-etal-2022-internet}, the authors describe a dialog focused system that is grounded with information retrieved from the internet at generation time.
To do this, their system generates a query based on its context, which they supply to a search engine such as Bing.
The documents that are retrieved by this search engine are then added to the context to generate the actual model response.

The real-time usage of the internet enables this approach to access the latest information available on the internet, similar to how a human would probably search for information.

Apart from the general concept of retrieving information and using it to improve the generation results, this approach has little in common with G-Retriever. While the authors of G-retriever, \citet{g-retriever}, mention in passing the option for a "Chat-with-your-graph" system, they do not explore this in their paper, and the resulting G-Retriever is not really a dialog focused system.

\subsection{Retrieval-Augmented Retriever Multimodal Language Modeling}

\subsection{Retrieval meets Long Context Large Language Models}



\section{Conclusion}

The field of RAG is a promising approach to improve the quality of generated text by incorporating information from retrieved documents while keeping resource requirements lower than scaling up the language model.
While most systems focus on unstructured text, such as news articles, scientific paper or Wikipedia, G-Retriever uses structured knowledge graphs as knowledge source.
This allows the system to make use of the information contained in the structure itself, as well as the text content, but also introduces the need for a more complex retrieval mechanism.

One possibility that is made available by RAG but seems under-explored in the literature, is the option to use the retrieved information to explain the model's reasoning.
This could provide users with more background information and enable them to fact-check and spot mistakes in the model's generation process.




\bibliography{custom}
\bibliographystyle{acl_natbib}

%\appendix
%
%\section{Example Appendix}
%\label{sec:appendix}
%
%This is an appendix.

\end{document}
