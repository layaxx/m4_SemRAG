% This must be in the first 5 lines to tell arXiv to use pdfLaTeX, which is strongly recommended.
\pdfoutput=1
% In particular, the hyperref package requires pdfLaTeX in order to break URLs across lines.

\documentclass[11pt]{article}

% Remove the "review" option to generate the final version.
\usepackage[review]{acl_mod}

% Standard package includes
\usepackage{times}
\usepackage{latexsym}

% For proper rendering and hyphenation of words containing Latin characters (including in bib files)
\usepackage[T1]{fontenc}
% For Vietnamese characters
% \usepackage[T5]{fontenc}
% See https://www.latex-project.org/help/documentation/encguide.pdf for other character sets

% This assumes your files are encoded as UTF8
\usepackage[utf8]{inputenc}

% This is not strictly necessary, and may be commented out,
% but it will improve the layout of the manuscript,
% and will typically save some space.
\usepackage{microtype}

% This package helps printing tables more nicely.
\usepackage{booktabs}

% Uncomment and modify for custom submission date
% \renewcommand{\subdate}{October 3, 2022}

%--------------------------------------------------------------------------------------------------
%--------------------------------------------------------------------------------------------------
% Modify title, author information, module name here

\def\paperTitle{G-Retriever in the context of the RAG research field}
\def\authors{Yannick Lang}
\def\matriculationNumber{1995498}
\def\moduleName{DS-SemRAG-M}

%--------------------------------------------------------------------------------------------------
%--------------------------------------------------------------------------------------------------


\title{\paperTitle}

\author{\authors \\
  Matriculation Number: \matriculationNumber \\
  \moduleName \\
  Faculty of Information Systems and Applied Computer Sciences\\
  Otto-Friedrich-University of Bamberg}

\begin{document}
\maketitle
\begin{abstract}
    The research field of Retrieval Augmented Generation (RAG) is a subfield of Natural Language Processing (NLP) that combines the strengths of retrieval-based and generation-based systems and has gained significant traction in the last years.
    This paper aims to give an overview over key concepts, contributions and approaches for RAG, focusing specifically on the graph-based G-Retriever model and its similarities with and difference to selected other models.
\end{abstract}


{\footnotesize
  \tableofcontents}

\section{Introduction}

While Large Language Models (LLMs) have been shown to perform well on a variety of Natural Language Processing (NLP) tasks, including question answering, they struggle with hallucinations and outdated or incomplete information.
LLMs are trained on large corpora of text data and retain a lot of information in their weights \cite{petroni-etal-2019-language}, but the amount of information is limited by the model size and information content is frozen during training.
This makes updating or changing knowledge expensive and time-consuming, as it requires retraining the model.

Retrieval-Augmented Generation (RAG) is a subfield of NLP that aims to address these issues by combining the strengths of retrieval-based and generation-based models.
Generally, this is achieved by retrieving relevant information from external knowledge sources and adding the retrieved information to the context in order to improve results of the generation process.

\section{Retrieval Augmented Generation}

This section will give an overview of the research field of Retrieval Augmented Generation (RAG), with a focus on its inception and key contributions leading up to the G-Retriever paper.

RAG is a subfield of Natural Language Processing that has gained traction since its inception in 2020. It combines the strengths of retrieval-based and generation-based models, aiming to improve the quality of generated text by incorporating information from retrieved documents.

The foundational work in this field was done by \cite{rag}, who introduced the RAG framework.
The main idea behind this paper was to allow pre-trained language models to access external knowledge sources during generation, thereby allowing the model to condition on information retrieved from external sources, in addition to  the knowledge embedded into its weight during the training process.
The authors show that such external knowledge sources can be updated, expanded, and refined independently of the model, which allows for more up-to-date and diverse information without the need to expend the resources associated with training large language models.

Because this approach works well for knowledge-intensive tasks, a common use case for RAG models is question answering.
Depending on the intended application, there are different conceivable knowledge sources.
Common choices include Wikipedia articles or scientific papers.
Some papers also make use of the internet as a knowledge source, which allows for more up-to-date information, but also introduces the risk of noise and misinformation.
This offers the opportunity to build on the immense effort that has already gone into optimizing search engines such as Bing or Google, instead of having to build or train a custom retrieval component.

\section{G-Retriever}

G-Retriever is a model that applies the RAG approach to text-based graphs.
It was introduced in 2024 by \cite{g-retriever}.


\bibliography{custom}
\bibliographystyle{acl_natbib}

%\appendix
%
%\section{Example Appendix}
%\label{sec:appendix}
%
%This is an appendix.

\end{document}
